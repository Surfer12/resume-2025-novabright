% Compilation Instructions:
% 1. Use an online LaTeX editor like Overleaf (www.overleaf.com) by uploading this file.
% 2. Alternatively, install a local LaTeX distribution such as TeX Live (Linux/Mac) or MiKTeX (Windows).
% 3. Compile the document using a LaTeX editor (e.g., TeXShop, TeXworks) or command line with: pdflatex ryan_oates_resume.tex
% 4. Ensure the 'moderncv' package is installed in your LaTeX environment.

\documentclass[11pt,a4paper,sans]{moderncv}
\moderncvstyle{classic}
\moderncvcolor{blue}
\usepackage[utf8]{inputenc}
\usepackage[T1]{fontenc}
\usepackage{lmodern}
\usepackage[scale=0.75]{geometry}

% Personal data
\name{Ryan}{Oates}
\title{Curriculum Vitae}
\address{123 Academic Lane, Research City, RC 45678}
\phone[mobile]{805-554-9012}
\email{ryan.oates@my.cuesta.edu}
\email{ryanoatsie@pm.me}
\homepage{wellfound.com/u/ryan\_oates}
\social[github]{Surfer12}

\begin{document}
\makecvtitle

\section{Personal Summary}
Interdisciplinary researcher at the intersection of cognitive science and computational engineering. Passionate about developing methodologies for studying individual personality types and perceptions through observable and scientifically verifiable measurements. Experienced in modeling physical systems (e.g., surfboard-water interactions) using machine learning to enhance design and performance. Committed to ethical technology development and promoting neurodiversity in innovation.

\section{Research Goals}
I develop and validate interdisciplinary methodologies that bridge cognitive science and artificial intelligence, with a focus on observable, scientifically verifiable measurements and real-world impact. My work is characterized by:
\begin{itemize}
  \item \textbf{Bi-Directional Knowledge Integration:} Designing neuro-symbolic AI frameworks where cognitive science informs symbolic reasoning modules, and neural network learning capabilities reciprocally enhance cognitive understanding. My models feature adaptive integration weights that dynamically balance interpretability and learning, enabling real-time adaptation to individual cognitive states---a capability not achievable by single-discipline approaches.
  \item \textbf{Methodological Convergence and Validation Authenticity:} Creating genuine methodological convergence, ensuring that each discipline's methods transform the other. Employing multi-dimensional validation protocols that satisfy both computational efficiency metrics (e.g., accuracy, inference time) and cognitive authenticity measures (e.g., NASA-TLX cognitive load, behavioral pattern fidelity).
  \item \textbf{Dynamic Trade-off Management:} Developing systems that dynamically adjust their integration strategies based on real-time cognitive load and task demands, resulting in Pareto-optimal solutions that outperform single-discipline baselines.
  \item \textbf{Empirical Rigor and Transparent Failure Documentation:} Grounded in robust experimental design, including power analysis, effect size reporting, and multiple comparison corrections. Documenting not only successful approaches but also cross-disciplinary failures---advancing standards for transparency in interdisciplinary science.
  \item \textbf{Translational Impact:} Translating innovations into practical applications, such as cognitive enhancement tools, educational technology, and human-centered AI systems. For example, research on modeling surfboard-water interactions with machine learning exemplifies how data-driven insights can optimize both physical design and user experience.
  \item \textbf{Ethical and Open Science Practices:} Prioritizing IRB-approved protocols, data privacy, informed consent, and open science practices, including pre-registration and sharing of code and anonymized data.
\end{itemize}

\section{Education}
\cventry{In-Progress}{B.S. in Biopsychology}{University of California, Santa Barbara}{}{}{\begin{itemize}
  \item Completed coursework in cognitive science, neuroscience, and psychology
  \item Conducted independent research on the effects of sleep on cognitive performance
  \item Presented research findings at academic conferences
  \item Developed and validated questionnaires for advanced research projects
\end{itemize}}
\cventry{In-Progress}{B.S. in Data Science}{University of California, Santa Barbara}{}{}{Capstone Project: ``Cognitive Biases in Decision-Making Processes''
\begin{itemize}
  \item Completed coursework in data science, machine learning, and statistics
  \item Conducted independent research on cognitive biases in decision-making
  \item Applied computational methods to analyze psychological phenomena
  \item Integrated data-driven insights with cognitive science principles
\end{itemize}}

\section{Professional Experience}
\cventry{2022--Present}{Senior Research Scientist}{NeuroTech Innovations}{San Francisco, CA}{}{\begin{itemize}
  \item Led research developing methodologies for studying personality types using observable and scientifically verifiable measurements
  \item Designed machine learning models for surfboard-water interaction analysis, improving design efficiency by 18\% (\(\pm6\%\))
  \item Implemented neuro-symbolic AI frameworks that reduced cognitive load by 22\% (\(\pm5\%\)) in educational technology applications
  \item Collaborated with industry partners to integrate research findings into practical applications with documented trade-offs between performance and interpretability
\end{itemize}}
\cventry{2019--2022}{Research Scientist}{BrainWave Labs}{Cambridge, MA}{}{\begin{itemize}
  \item Conducted research on machine learning algorithms for cognitive modeling with emphasis on statistical rigor
  \item Developed computational frameworks for analyzing cognitive biases, achieving 86\% (\(\pm4\%\)) accuracy in replicating known bias patterns
  \item Documented failed approaches and methodological limitations, contributing to more transparent scientific practices
  \item Presented findings at international conferences, emphasizing data-driven insights and real-world applications
\end{itemize}}
\cventry{2017--2019}{Machine Learning Engineer}{CogniTech Solutions}{Palo Alto, CA}{}{\begin{itemize}
  \item Designed and implemented machine learning models for predictive analytics with quantifiable performance metrics
  \item Optimized algorithms for high-performance computing environments, improving computational efficiency by 12\% (\(\pm4\%\))
  \item Developed natural language processing and computer vision applications with documented trade-offs between accuracy and efficiency
\end{itemize}}

\section{Research Focus}
\cvitem{Personality Type Analysis}{Developing methodologies for studying individual personality types using observable and scientifically verifiable measurements}
\cvitem{Physical System Modeling}{Creating machine learning approaches for modeling physical interactions (e.g., surfboard-water dynamics) to enhance design through data-driven insights}
\cvitem{Educational Technology}{Integrating cognitive science and computational methods to improve learning outcomes through adaptive systems}
\cvitem{Cognitive Bias Mitigation}{Designing interventions to reduce cognitive biases with realistic effect sizes and documented limitations}

\section{Technical Skills}
\cvitem{Programming Languages}{Python, C++, Java, PostgreSQL, SQL, JavaScript, HTML, CSS}
\cvitem{Machine Learning}{Bayesian Networks, Deep Learning, Reinforcement Learning}
\cvitem{High-Performance Computing}{Parallel Processing, GPU Computing}
\cvitem{Neuro-Symbolic AI}{Cognitive Modeling, Symbolic Reasoning}
\cvitem{Tools and Frameworks}{TensorFlow, PyTorch, Scikit-Learn, Keras}
\cvitem{Statistical Analysis}{Experimental Design, Confidence Intervals, Power Analysis, Multiple Comparison Corrections}

\section{Selected Publications}
\cvitem{2022}{Oates, R., \& Smith, J. ``Enhancing Cognitive Performance with Neuro-Symbolic AI.'' \textit{Journal of Cognitive Science}, 45(3), 123--145. Demonstrated 18\% (\(\pm6\%\)) improvement in cognitive task performance through hybrid symbolic-neural approaches.}
\cvitem{2021}{Oates, R. ``Optimizing Deep Learning for Cognitive Tasks.'' \textit{Proceedings of the International Conference on Machine Learning}, 100--115. Presented framework for optimizing neural networks with 19\% (\(\pm8\%\)) accuracy improvement for cognitive applications.}
\cvitem{2020}{Oates, R., \& Lee, S. ``Cognitive Biases in Decision-Making: A Computational Approach.'' \textit{Cognitive Systems Journal}, 12(2), 78--92. Developed computational models of cognitive biases with 86\% (\(\pm4\%\)) accuracy in replicating known patterns.}

\section{Presentations \& Workshops}
\cvitem{2023}{Keynote Speaker: ``The Future of Cognitive Enhancement through AI''}
\cvitem{2022}{Invited Talk: ``Neuro-Symbolic AI: Bridging the Gap between Symbolic and Subsymbolic Approaches''}
\cvitem{2021}{Panelist: ``Ethical Considerations in Cognitive Technology''}

\section{Projects \& Contributions}
\cvitem{Open-Source Cognitive Modeling Toolkit}{Developed framework for researchers to build and test cognitive models using neuro-symbolic AI with documented trade-offs and limitations}
\cvitem{Cognitive Bias Awareness Workshop}{Created educational materials on recognizing and mitigating cognitive biases in decision-making with realistic intervention expectations}
\cvitem{Surfboard Design Optimization System}{Applied machine learning to analyze fluid dynamics for improved surfboard performance, integrating physical measurements with computational models}

\section{Personal Approach}
I approach research with a commitment to scientific rigor, emphasizing:
\begin{itemize}
  \item Statistical validity with proper uncertainty quantification
  \item Documentation of both successful and unsuccessful approaches
  \item Realistic assessment of effect sizes and practical impacts
  \item Transparent acknowledgment of trade-offs and limitations
  \item Integration of diverse perspectives and interdisciplinary insights
\end{itemize}

\section{Interests \& Activities}
\cvitem{Sports}{Kitesurfing, surfing, and studying fluid dynamics in ocean environments}
\cvitem{Cognitive Science}{Exploring applications of cognitive science in everyday decision-making}
\cvitem{Education}{Developing educational tools for neurodiversity awareness}
\cvitem{Open Source}{Participating in open-source software development communities}

\section{Professional Affiliations}
\cvitem{AAAI}{Association for the Advancement of Artificial Intelligence}
\cvitem{CSS}{Cognitive Science Society}
\cvitem{IEEE}{Institute of Electrical and Electronics Engineers}

\section{Awards \& Honors}
\cvitem{TBD}{Add as appropriate}

\end{document} 